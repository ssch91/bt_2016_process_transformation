\chapter{Einleitung}

\section{Motivation}

Inhalt und Ziel dieser Arbeit ist es, einen Kreditorenworkflow am konkreten Beispiel der GLC Glücksburg Consulting AG zu analysieren und mit Blick auf bestehende Stärken und Schwächen, z.B. hinsichtlich der Leistungsfähigkeit, neu zu konzeptionieren und dabei immanententes oder extrinsisches Verbesserungspotential zu erkennen und dessen Implementierung zu planen.\\
Der Wunsch nach einem derartigen Projekt wurde an den Autor bereits vor mehreren Jahren herangetragen und entstand laut den Projekttreibern aus der Fehleranfälligkeit und aufgrund der immensen mittleren Prozessdurchlaufzeit. \\
Anders als in den verwendeten Quellen ist die Motivation für das Projekt, dessen Definitionsphase diese Arbeit darstellt, also primär auf interne Ambitionen gestützt.
Ein entscheidender Unterschied zu anderen, in dieser Art und Weise durchgeführten Projekten ist, dass der Nutzen des Prozesses nur intern generiert wird und der Prozessoutput keine verrechenbare Leistung gegenüber potentiellen Kunden ist.\\


-> häufig innovations- oder projektgetrieben
-> Kundenbezug

In unserem Fall aber einfach nicht, sondern Fehlerhäufigkeit und Performanceschwierigkeiten

\section{Methodisches Vorgehen}


-> Anwendung von Methoden der Geschäftsarchitektur nur strukturell begrenzt aufgrund des Umfanges

-> Erhebung Ist-Zustand nur funktional relevant, da ein Soll-Zustand weitgehend gemäß des Idealzustands (Leuchtturm) erarbeitet werden soll
aber: rechtliche Anforderungen!

-> Struktur der Methoden passt, inhaltlich aber zu weit gefasst

-> BPMN und UML

-> Unterstützungs- und Leisungsprozess, da Buchhaltung als Service eine von der GLC angebotene Dienstleistung ist
Anforderungen daran sind in Quasar Enterprise festgehalten (S. 75)
Teil der QM-Vorgaben nach DIN EN ISO 9001 Kapitel 4.2.2 "Wechselwirkungen der Prozesse" 

-> Bestandsaufnahme in Teil 1, Entwicklung von Grundlagen des Performance Measurement, Entwicklung des Leuchtturms in Teil 2

\subsection{}