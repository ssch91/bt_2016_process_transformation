\chapter{Einleitung}

\section{Motivation}

Inhalt und Ziel dieser Arbeit ist es, einen Kreditorenworkflow am konkreten Beispiel der GLC Glücksburg Consulting AG zu analysieren und mit Blick auf bestehende Stärken und Schwächen, z.B. hinsichtlich der Leistungsfähigkeit, neu zu konzeptionieren und dabei immanententes oder extrinsisches Verbesserungspotential zu erkennen und dessen Implementierung zu planen.\\
Der Wunsch nach einem derartigen Projekt wurde an den Autor bereits vor mehreren Jahren herangetragen und entstand laut den Projekttreibern aus der Fehleranfälligkeit und aufgrund der immensen mittleren Prozessdurchlaufzeit. \\
Anders als in den verwendeten Quellen ist die Motivation für das Projekt, dessen Definitionsphase diese Arbeit darstellt, also primär auf interne Ambitionen gestützt.
Ein entscheidender Unterschied zu anderen, in dieser Art und Weise durchgeführten Projekten ist dahingehend, dass der Nutzen des Prozesses nur internen Akteuren gegenüber generiert wird und der Prozessoutput keine verrechenbare Leistung gegenüber potentiellen Kunden darstellt.\\
Hierbei ist nichtsdestotrotz die Nutzung innovationsinduzierten Technologiepotentials ein Rahmenaspekt, den insbesondere die IT-Bereichsleitung berücksichtigt wissen will, da interne Projekte und Erfahrungen als Lernplattform und Maßstab betrachtet werden. \\


-> häufig innovations- oder projektgetrieben
-> Kundenbezug

In unserem Fall aber einfach nicht, sondern Fehlerhäufigkeit und Performanceschwierigkeiten

\section{Methodisches Vorgehen}
\subsection{Methodenauswahl}

Das Vorgehen dieser Arbeit stützt sich auf Quellen aus drei Themenbereichen.

\begin{enumerate}

\item{Vorlesungsinhalte}
\\ Erklärung 1

\item{Rahmenwerke zur Gestaltung von Unternehmensarchitekturen}
\\ Erklärung 2

\item{Normen, Gesetze und Verordnungen}
\\ Erklärung 3
\end{enumerate}


-> TOGAF und QUASAR = Friss oder Stirb \\
-> Arbeit ist zu kurz, um hier ins Detail zu gehen\\
-> Methoden aus IT-Orga und PM und Softwaretechnik Vorlesungen, weil sie bekannt sind\\
-> ISO 9000-Reihe als gängiges Rahmenwerk wenn's um Prozesse geht\\
\subsection{Teil I: Analyse}

Am Anfang der Arbeit steht eine Aufnahme des Ist-Zustands von besagtem Workflow. Dieser muss zunächst eingeordnet werden, um ihn konkret analysieren zu können. 
Die Auswertung des Prozessablaufs auf Basis von Interviews und bestehender Prozessdokumentation ist die Grundlage für Analyse. 
In dieser werden Stärken und Schwächen des Ist-Zustands identifiziert und in Key-Performance-Indicators überführt, die ein zentraler Bestandteil dieser Arbeit sind, da sie die Grundlage für das Performance-Measurement sind, welcher die Vergleichbarkeit zum Soll-Konzept in Teil II darstellt. 
Die abschließende Einordnung in ein prozessportfolio ist der Ausgangspunkt für die Neukonzeption in Teil II.

\subsection{Teil II: Neukonzeption}

Die Neukonzeption des Workflows orientiert sich methodisch an Rahmenwerk der Open Group: The Open Group Architecture Framework. 
Zur konkreten Umsetzung wird Quasar Enterprise als Rahmenwerk herangezogen. 
Das von TOGAF strukturell vorgegebene Vorgehen, das hauptsächlich zu erzeugende und auszuwertende Artefakte definiert, wird insofern durch Quasar Enterprise ergänzt, als das dieses Methoden und Regeln zur Umsetzung vorschlägt. 
Vereinfacht lässt sich konstatieren, dass TOGAF definiert, was getan wird, und Quasar Enterprise erläutert, wie es getan wird. 
Die detaillierte Beschreibung des Vorgehens folgt in Teil II.\\
Mittels dieser Rahmenwerke werden ein Soll-Prozess und eine dafür notwendige Architektur entwickelt, die perspektivisch die Grundlage für eine Umsetzung bilden sollen.\\
Nachdem dieses Basis entwickelt ist, wird mittels Anwendbarkeitsprüfung der in Teil I entwickelten KPI's die Vergleichbarkeit hergestellt und ggf. weitere KPI's entwickelt oder bestehende KPI's verändert, um die Adäquanz gegenüber der neu entwickelten Architektur herzustellen.

-> Anwendung von Methoden der Geschäftsarchitektur nur strukturell begrenzt aufgrund des Umfanges

-> Erhebung Ist-Zustand nur funktional relevant, da ein Soll-Zustand weitgehend gemäß des Idealzustands (Leuchtturm) erarbeitet werden soll
aber: rechtliche Anforderungen!

-> Struktur der Methoden passt, inhaltlich aber zu weit gefasst

-> BPMN und UML

-> Unterstützungs- und Leisungsprozess, da Buchhaltung als Service eine von der GLC angebotene Dienstleistung ist
Anforderungen daran sind in Quasar Enterprise festgehalten (S. 75)
Teil der QM-Vorgaben nach DIN EN ISO 9001 Kapitel 4.2.2 "Wechselwirkungen der Prozesse" 

-> Bestandsaufnahme in Teil 1, Entwicklung von Grundlagen des Performance Measurement, Entwicklung des Leuchtturms in Teil 2

\subsection{}