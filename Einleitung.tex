\chapter{Einleitung}

\section{Motivation}

-> häufig innovations- oder projektgetrieben
-> Kundenbezug

In unserem Fall aber einfach nicht, sondern Fehlerhäufigkeit und Performanceschwierigkeiten

\section{Methodisches Vorgehen}


-> Anwendung von Methoden der Geschäftsarchitektur nur strukturell begrenzt aufgrund des Umfanges

-> Erhebung Ist-Zustand nur funktional relevant, da ein Soll-Zustand weitgehend gemäß des Idealzustands (Leuchtturm) erarbeitet werden soll
aber: rechtliche Anforderungen!

-> Struktur der Methoden passt, inhaltlich aber zu weit gefasst

-> BPMN und UML

-> Unterstützungs- und Leisungsprozess, da Buchhaltung als Service eine von der GLC angebotene Dienstleistung ist
Anforderungen daran sind in Quasar Enterprise festgehalten (S. 75)
Teil der QM-Vorgaben nach DIN EN ISO 9001 Kapitel 4.2.2 "Wechselwirkungen der Prozesse" 