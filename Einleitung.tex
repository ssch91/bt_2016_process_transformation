\chapter{Einleitung}

\section{Motivation}

Inhalt und Ziel dieser Arbeit ist es, einen Kreditorenworkflow am konkreten Beispiel der GLC Glücksburg Consulting AG zu analysieren und mit Blick auf bestehende Stärken und Schwächen, z.B. hinsichtlich der Leistungsfähigkeit, neu zu konzeptionieren und dabei inhärentes oder mit strukturellen Veränderungen verbundenes Verbesserungspotential zu erkennen und dessen Implementierung zu planen.\\
Der Wunsch nach einem derartigen Projekt wurde an den Autor bereits vor mehreren Jahren herangetragen und entstand laut den Projekttreibern immer wieder im Zusammenhang mit dem Prozess auftretenden Komplikationen. \\
Demnach basiert die Projektidee primär auf internen Ambitionen, zumal Inhalte der Finanzbuchhaltung in der Regel internen Anforderungen entsprechen müssen.
%Die Kreditorenbuchhaltung hat jedoch im konkreten Fall der \firma die Besonderheit, dass diese auch als Outsourcing-Leistung auf dem Markt angeboten wird. 
Dieser Umstand verdeutlicht den Wunsch betroffenener Mitarbeiter sowie des involvierten Managements, die Tätigkeiten auf Basis eines unproblematischen und performanten Prozess abwickeln zu können.
Ein weiterer Rahmenspekt ist die Nutzung die Nutzung innovationsinduzierten Technologiepotentials, das insbesondere die IT-Bereichsleitung berücksichtigt wissen möchte. 
Die internen Projekte und Erfahrungen der \firma dienen als Lernplattform und bilden häufig den Maßstab für zukünftige Kundenprozesse.
Häufig sind es interne Ideen und Prototypen, die das angebotene Leistungsspektrum der \firma erweitern und so im Endeffekt in IT-Landschaften von Kunden zum Einsatz kommen.\\
Diese Arbeit stellt die Planungs-, Definitions- und Entwurfsphase\footnote{Vgl. Vorlesungsfolien Prof. Dr. Joachim Sauer im Fach Softwaretechnik (2016), im Falle einer klassischen Einordnung in ein Wasserfallmodell wie zunächst beschrieben in \cite{united1956symposium} und von Prof. Dr. Sauer adaptiert sind diese Phasen vor der Implementation. } des Projekts dar.
%Anders als in den verwendeten Quellen ist die Motivation für das Projekt, dessen Definitionsphase diese Arbeit darstellt, also primär auf interne Ambitionen gestützt.
%Ein entscheidender Unterschied zu anderen, in dieser Art und Weise durchgeführten Projekten ist dahingehend, dass der Nutzen des Prozesses nur internen Akteuren gegenüber generiert wird und der Prozessoutput keine verrechenbare Leistung gegenüber potentiellen Kunden darstellt.\\




\section{Methodisches Vorgehen}
\subsection{Methodenauswahl}

Das Vorgehen dieser Arbeit stützt sich auf Quellen aus drei Themenbereichen.

\begin{enumerate}



\item{Rahmenwerke zur Gestaltung von Unternehmensarchitekturen}
\\ Als Hauptbezug dienen zwei Rahmenwerke für Entwurf, Planung, Implementierung und Wartung von Unternehmensarchitekturen. 
Primär orientiert sich der zweite Teil dieser Arbeit an The Open Group Architecture Framework (TOGAF), welches aufgrund seiner Bekanntheit und Empfehlung durch den Erstprüfer dieser Arbeit zu Gunsten des Umfangs keiner weiteren Methodenauswahl unterworfen wird. \\
TOGAF wird durch Quasar Enterprise ergänzt, welches sich konkrekt auf derartige Rahmenwerke bezieht und Methoden, Verfahren und Regeln zur Nutzung dieser bereitstellt. 
Die Nutzung von Quasar Enterprise erfolgt aufgrund des Vorlesungsbezugs in Softwaretechnik zu Quasar, sowie ebenfalls auf Empfehlung des Erstprüfers.

\item{Vorlesungsinhalte}
\\ Speziell zur Prozessanalyse und Visualisierung bzw. Modellierung diesbezüglicher Bereiche werden Methoden und Verfahren aus den Vorlesungen Softwaretechnik und Geschäftsprozessmodellierung und QM verwendet, z.B. die Unified Modelling Language (UML) und Business Process Model and Notation (BPMN). 
Diese werden zur Analyse und zum Entwurf von TOGAF und Quasar Enterprise vorgeschlagen.

\item{Normen, Gesetze und Verordnungen}
\\ Im Zusammenhang mit Rechnungsverwaltung und -buchung gibt es gesetzliche Vorschriften, die bei der Konzeption von Prozessstrukturen zu beachten sind.
Insbesondere ist dies die 2014 veröffentlichte Verordnung zur ordnungsgemäßen Führung und Aufbewahrung von Büchern, Aufzeichnungen und Unterlagen in elektronischer Form sowie zum Datenzugriff (GoBD), die zusammenfassend die damit obsoleten Grundsätze zum Datenzugriff und zur Prüfbarkeit digitaler Unterlagen (GDPdU) und Grundsätze ordnungsgemäßer DV-gestützter Buchführungssysteme (GoBS) ablöst.\\
Zusätzlich wird die ISO 9000 Reihe herangezogen, unter deren Gesichtspunkten bereits früher von einem Dienstleiter Prozesse bei der GLC begutachtet wurden.
\end{enumerate}

\subsection{Teil I: Analyse}

Am Anfang der Arbeit steht eine Aufnahme des Ist-Zustands von besagtem Workflow. Dieser wird zunächst eingeordnet, um ihn konkret analysieren zu können. 
Die Auswertung des Prozessablaufs auf Basis von Interviews und bestehender Prozessdokumentation ist die Grundlage für Analyse. 
In dieser werden Stärken und Schwächen des Ist-Zustands identifiziert und in Key-Performance-Indicators überführt, die ein zentraler Bestandteil dieser Arbeit sind, da sie die Grundlage für das Performance-Measurement sind, welches die Vergleichbarkeit zum Soll-Konzept in Teil III herstellt.
Die Erhebung des Ist-Zustandes dient in diesem Rahmen weitestgehend zunächst der Anforderungsanalyse an die Neukonzeption. 
Da sich die Motivation bzw. Notwendigkeit der Neuausrichtung des Workflows aus dieser Analyse ergeben, stellt diese einen maßgeblichen Teil der Arbeit dar und erfolgt deshalb vorgelagert und nicht als Teil einer Iteration in einem der genannten Frameworks.
Dieses Vorgehen ist zwar möglich, misst diesem Teil aber nicht nötige Relevanz bei.
Auf die strukturell zu identifizierenden bestehenden Elemente wird allerdings im Zuge der Neukonzeption zurückgegriffen.
%Die abschließende Einordnung in ein Prozessportfolio ist der Ausgangspunkt für die Neukonzeption in Teil II.

\subsection{Teil II: Neukonzeption}

Die Neukonzeption des Workflows orientiert sich methodisch an Rahmenwerk der Open Group: The Open Group Architecture Framework. 
Zur konkreten Umsetzung wird Quasar Enterprise als Rahmenwerk herangezogen. 
Das von TOGAF strukturell vorgegebene Vorgehen, das hauptsächlich zu erzeugende und auszuwertende Artefakte definiert, wird insofern durch Quasar Enterprise ergänzt, als dass dieses Methoden und Regeln zur Umsetzung vorschlägt. 
Vereinfacht lässt sich konstatieren, dass TOGAF definiert, was getan wird, und Quasar Enterprise erläutert, wie es getan wird. 
Die detaillierte Beschreibung des Vorgehens folgt in Teil II.\\
Mittels dieser Rahmenwerke werden ein Soll-Prozess und eine dafür notwendige Architektur entwickelt, die perspektivisch die Grundlage für eine Umsetzung bilden sollen.\\
Nachdem dieses Basis entwickelt ist, wird mittels Anwendbarkeitsprüfung der in Teil I entwickelten KPI's die Vergleichbarkeit hergestellt und ggf. weitere KPI's entwickelt oder bestehende KPI's verändert, um die Adäquanz gegenüber der neu entwickelten Architektur herzustellen.\\

%-> Anwendung von Methoden der Geschäftsarchitektur nur strukturell begrenzt aufgrund des Umfanges

%-> Erhebung Ist-Zustand nur funktional relevant, da ein Soll-Zustand weitgehend gemäß des Idealzustands (Leuchtturm) erarbeitet werden soll
%aber: rechtliche Anforderungen!

%-> Struktur der Methoden passt, inhaltlich aber zu weit gefasst

%-> BPMN und UML

%-> Unterstützungs- und Leisungsprozess, da Buchhaltung als Service eine von der GLC angebotene Dienstleistung ist
%$\rightarrow$Anforderungen daran sind in Quasar Enterprise festgehalten (S. 75)\\
%$\rightarrow$Teil der QM-Vorgaben nach DIN EN ISO 9001 Kapitel 4.2.2 "Wechselwirkungen der Prozesse" 

%-> Bestandsaufnahme in Teil 1, Entwicklung von Grundlagen des Performance Measurement, Entwicklung des Leuchtturms in Teil 2

