\section{Simulationsprobleme}

Um Aussagen über Modelle treffen zu können müssen viele Experimente mit gleichen, aber auch mit verschiedenen Parameterkombinationen durchgeführt werden. Jede dieser Parameterkombination wird in einer sog. Replikation benutzt. Diese Replikationen müssen mehrmals ausgeführt werden, da die Durchführung stochastisch sein kann. Anders als bei komplexitätstheoretisch schwierigen Problemen ergibt sich bei Simulationsalgorithmen der größte Teil der Laufzeit- und Platzkomplexität anhand der Modellgröße, der geforderten Replikationen und der Simulationslaufzeit. Als Folge der sich oft wiederholenden Berechnungsarten ist es möglich mit geringen Performanceverbesserungen bei einzelnen Algorithmen eine spürbare Verbesserung der Laufzeit des Gesamtsystems zu bewirken. 

Durch die große Anzahl an Modellen und das unvorhersehbare Verhalten bei verschiedenen Parameterkombinationen in Experimenten, dazu den vielen verschiedenen Algorithmen um Experimente auf Modellen auszuführen kann meistens nicht vorhergesagt werden, welcher Algorithmus die besten Ergebnisse erzielen würde. Aus diesem Grund ist eine Analyse der Algorithmenportfolios und ihrer Theorie im Zusammenhang mit Simulationsproblemen von Interesse, damit Leistungsänderungen durch Anwendung von Portfolioansätzen besser einschätzt werden können.

Im folgenden werde ich mich bei der Betrachtung von Algorithmenportfolios im Zusammenhang mit Simulationsproblemen meistens auf zelluläre Automaten beziehen, da sich mit ihnen die meisten Sachverhalte einfach und anschaulich beschreiben lassen. Weitere Informationen bzgl. zellulärer Automaten und Simulationsalgorithmen bietet \cite{himmelspach09}.
