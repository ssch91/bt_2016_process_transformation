\usepackage{ifpdf}

% % Um dieses Document sowohl fuer latex als auch fuer pdflatex
% % kompatibel zu machen, definieren wir uns die Variable ifpdf
%%% check whether we are running pdflatex
%\newif\ifpdf
%\ifx\pdfoutput\undefined
%\pdffalse 		% we are not running pdflatex
%\else
%\pdfoutput=1 		% we are running pdflatex
%%das muessen wir nicht unbedingt uebernehmen
%\pdfcompresslevel=9    % compression level for text and image;
%\pdftrue 
%\fi

\ifpdf
%%das kann man benutzen, wenn man andere Formate benutzen will
%\DeclareGraphicsExtensions{{.pdf}}   %Endung der Grafiken, wenn nicht pdf
% die folgenden Angaben sind im PDF unter Datei | Dokumenteigenschaften 
% in Acrobat / Acrobat Reader sichtbar
% Aendern Sie bitte die Daten, wo noetig!
\usepackage[%
	pdftitle={\dctitle},
	pdfauthor={\dcauthorsurname\ \dcauthorname},
	pdfsubject={\dcpdfsubject}, % optional
	pdfkeywords={\dckeydea, \dckeydeb, \dckeydec, \dckeyded},
	pdfpagemode=UseOutlines,
  colorlinks=true,					% bitte nicht ändern!
	linkcolor=black,					% bitte nicht ändern!
	filecolor=black,					% bitte nicht ändern!
	urlcolor=black,						% bitte nicht ändern!
	citecolor=black,					% bitte nicht ändern!
	pdftex=true,              % bitte nicht ändern!
	plainpages=false,         % bitte nicht ändern!
	hypertexnames=false,      % bitte nicht ändern!
	pdfpagelabels=true,       % bitte nicht ändern!
	hyperindex=true]{hyperref}% bitte nicht ändern!
\else
  % hier kann mann eventuelle Befehle umdefinieren
  % die nur für pdfLaTeX vorgesehen sind
  % und das richtige Kompilieren durch den normalen LaTeX verhindern
	\newcommand{\texorpdfstring}[2]{#2}
\fi