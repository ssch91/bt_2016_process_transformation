%-Eingabe der Metadaten des Titelblattes--------------------------

%-Daten des Autors / Authors Data---------------------------------

\newcommand{\dcauthorpre}{Herr/Frau} %ggf Titel
\newcommand{\dcauthorsurname}{Nachname} 
\newcommand{\dcauthorname}{Vorname} 
\newcommand{\dcauthoradd}{Irgendwann in Irgendwo} %geboren am 
\newcommand{\dcauthorno}{Matrikelnummer} %Matrikelnummer

%-Titel und Untertitel / Title and subtitle-----------------------

\newcommand{\dctitle}{Der Titel der Arbeit} 
\newcommand{\dcsubtitle}{~}  
% Falls dcsubtitle NICHT verwendet werden soll, {\dcsubtitle}{~} eingeben.

%-Eingabe der Gutachternamen / Names of the approvals-------------

\newcommand{\dcapprovala}{Dr.-Ing. Jan Himmelspach} 
\newcommand{\dcapprovalb}{Zweitgutachter} 
\newcommand{\dcapprovalc}{} 

%-Eingabe der Betreuernamen / Name of the advisors ---------------

\newcommand{\dcada}{Betrieblicher Betreuer}

%-Information zur Universitaet------------------------------------

\newcommand{\dcdegree}{Bachlor of Science\\(BSc)} 
\newcommand{\dcsubject}{Informatik} 
\newcommand{\dcfaculty}{Fachbereich Informatik}
\newcommand{\dcuniversity}{NORDAKADEMIE}
\newcommand{\dcdean}{}
\newcommand{\dcpresident}{Prof. Dr. Behringer} %???

%-Pruefungsdaten: eingereicht und mdl. Pruefung-------------------
%-data of submission and oral exam--------------------------------

\newcommand{\dcdatesubmitted}{31. Januar 2011} %auch wenn nicht auf dem Titelblatt, bitte erfüllen!
\newcommand{\dcdateexam}{n.n.} 

%-CR Klassifikation / cr classification --------------------------
%siehe: http://www.acm.org/class/1998/
\newcommand{\dccrclassification}{I.6.8[Simulation and Modelling]:Types of Simulation---Discrete event, Distributed, Parallel;}

%-deutsche Schlagwoerter / german keywords------------------------

\newcommand{\dckeydea}{Schlagwort 1}
\newcommand{\dckeydeb}{Schlagwort 2}
\newcommand{\dckeydec}{Schlagwort 3}
\newcommand{\dckeyded}{Schlagwort 4}

% Folgende Zeile bitte nicht aendern!
\newcommand{\dckeywordsde}{\vfill \raggedright {\textbf{Schlagw\"orter:}}\\ \dckeydea, \dckeydeb, \dckeydec, \dckeyded \\}

%-englische Schlagwoerter / english keywords----------------------

\newcommand{\dckeyena}{keyword 1}
\newcommand{\dckeyenb}{keyword 2}
\newcommand{\dckeyenc}{keyword 3}
\newcommand{\dckeyend}{keyword 4}

% Folgende Zeile bitte nicht aendern!
\newcommand{\dckeywordsen}{\vfill \raggedright {\textbf{Keywords:}}\\ \dckeyena, \dckeyenb, \dckeyenc, \dckeyend \\}
%\newcommand{\dccrandkeywordsen}{\vfill \raggedright { \small {\textbf{CR--Classification (1998)::}}\\ \dccrclassification \\}\vspace{4mm}{\textbf{Keywords:}}\\ \dckeyena, \dckeyenb, \dckeyenc, \dckeyend \\}
%\raggedright removed from line below (after \vfill)
\newcommand{\dccrandkeywordsen}{\vfill { \small \noindent {\textbf{CR--Classification (1998)::}}\\ \dccrclassification \\}\vspace{4mm}{\textbf{Keywords:}}\\ \dckeyena, \dckeyenb, \dckeyenc, \dckeyend \\}

\newcommand{\dcpdfsubject}{Bachelorarbeit}
