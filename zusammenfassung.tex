\chapter{Zusammenfassung}

In der vorliegenden Arbeit wurden Algorithmenportfolios und ihre Anwendbarkeit auf Simulationsprobleme betrachtet. Sie bieten einen interessanten Ansatz mit den Schwierigkeiten des von Rice vorgestellten Algorithm Selection Problems umzugehen. 

Die Konstruktion von Algorithmenportfolios ist ein in der Literatur wenig beachtetes Thema, die meisten Arbeiten betrachten von vornherein nur kleine Algorithmenmengen (2-3) und erstellen daraus ihre Portfolios.

Die große Komplexität von Simulationsproblemen macht es schwierig Algorithmenportfolios allgemein im Zusammenhang mit ihnen zu beurteilen. Restartstrategien ermöglichen mitunter gute Leistungssteigerungen, sie sind jedoch für Simulationsalgorithmen nur bedingt anwendbar. Die Einordnung von Simulationsalgorithmen in die oft betrachteten Kategorien der Entscheidungs- und Optimierungsprobleme lässt sich nicht durchführen, wodurch bisherige Ergebnisse nicht direkt übertragen und immer kritisch begutachtet werden müssen. 

Die Kooperation von Algorithmen innerhalb eines Portfolios und die Ausnutzung von Problemstrukturen ist ein interessanter Ansatz zur Verbesserung der Leistung. Für eine zweckmäßige Umsetzung und Evaluierung dieses Ansatzes müssten Portfolios individuell betrachtet werden, was einer automatisierten Konstruktion effektiver Portfolios widerspricht. 

Die drei verschiedenen Ausführungsarten haben ihre jeweiligen Vor- und Nachteile. Die einzelne Ausführung eines Algorithmus wäre für Simulationsprobleme gut geeignet, sofern genügend Leistungsdaten gesammelt sind und/oder alle Ausführungen in die Simulationsergebnisse mit einfließen sollen und kein aufwändiger, länger zu berechnender Verlauf ignoriert werden darf. Ist dies nicht der Fall bietet die verschränkte Ausführung womöglich bessere Aussichten. Eine parallele Ausführung der Algorithmen scheint aufgrund der vielfach vorhandenen Parallelität innerhalb eines Simulationsystem nicht zweckmäßig zu sein. Zudem müssten, sofern jeder Verlauf berücksichtigt werden muss, alle Algorithmen beendet werden!

Die Offline- oder Onlineentwicklung für Simulationsprobleme zu nutzen scheint sinnvoll, eine statische Entwicklung wäre zu unflexibel gegenüber neuen Problemarten.

Insgesamt bieten Algorithmenportfolios interessante Möglichkeiten die Performance von Simulationen zu beeinflussen. Aufgrund der großen Anzahl an Variationen und möglichen Problemen müsste sich jedoch für jeden Einsatz überlegt werden welchen Typ Portfolio man konstruieren und benutzen möchte. Die wichtige Frage, ob signifikante Leistungssteigerungen mit ihrer Hilfe erzielt werden können, kann im Allgemeinen noch nicht beantwortet werden.


